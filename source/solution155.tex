\begin{task}{155}
Пусть $f,\,g,\,h$ -- неубывающие функции из $\mathbb{R}^+$ в $\mathbb{R}^+$. Пусть $n \to \infty$. Верно ли, что если $f(n)=O(g(n))$ и $g(n)=o(h(n))$, то обязательно $f(n)=o(h(n))$? Если верно, то обоснуйте, опираясь исключительно на определения. Если не верно в общем случае, то приведите соответствующий контрпример.
\end{task}

\begin{solution}
Вспомним определения $o$-малого и $O$-большого: \par
Пусть $f(x)$ и $g(x)$ — две функции, определенные в $U_\varepsilon(x_0)$ и $\lim_{x\to x_0} g(x) \neq 0$. Тогда говорят, что: \newline
$f = O(g)$, при $x \to x_0$, если $\exists C > 0 : \forall x \in U_\varepsilon(x_0) \Rightarrow |f(x)| \leq C|g(x)|$.\newline
$f = o(g)$, при $x \to x_0$, если $\forall c > 0,\, \exists \delta > 0 : \forall x \in U_\delta(x_0) \Rightarrow |f(x)| < c|g(x)|$.
\newline \newline
Теперь, с учетом того, что $f,\,g,\,h$ -- неубывающие функции и $x_0 = + \infty$:\par
(1) $\exists C > 0$ и $\exists x_1 \neq + \infty : \forall x > x_1 \Rightarrow |f(x)| \leq C|g(x)|$.\par
(2) $\forall c > 0 \exists x_2 \neq + \infty : \forall x > x_2 \Rightarrow |g(x)| < c|h(x)|$.\newline
Возьмем в (2) $c = C$. Пусть также $x_3(c) = \max(x_1, x_2(c))$. Тогда выполнено: $\forall x > x_3: |f(x)| \leq C|g(x)| < C^2|h(x)|$. \newline
Значит $\forall c > 0 \exists x_3(c) \neq + \infty : \forall x > x_3 \Rightarrow |f(x)| < C'|h(x)|$ (тут $C' = C^2$).\newline
Отсюда $f(n)=o(h(n))$.
\end{solution}