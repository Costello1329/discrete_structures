\begin{task}{157}
Пусть в двудольном графе $G$ с долями $X$ и $Y$ существует совершенное паросочетание, и пусть $\operatorname{deg} v \geq t$ для каждой вершины $v \in X$. Докажите, что в $G$ найдутся не менее $t!$ различных совершенных паросочетаний.
\end{task}

\begin{solution}
В $G$ есть совершенное пароочетание по условию. Значит, доли $X$ и $Y$ имеют одинаковую мощность. Помимо этого, выполняются условия Холла: $\forall \ U \subset X : |N(U)| \geq |U|$. Будем использовать конструктивный метод доказательства, а именно --- индукцию. \\
База: $|X| = |Y| = 1$: $\forall \ u \in X: \operatorname{deg}(v) = 1, \exists \ !$ \text{паросочетание}. \\
Чтобы доказать переход, придется рассмотреть два случая:
\begin{enumerate}
    \item $\exists \text{ непустое } \ U_1 \subset X \text{ (включение строгое)} : |N(U_1)| = |U_1|$. Далее, докажем существование совершенного паросочетания не только в $G$, но и в $G'(V'_1, V'_2)$, где $V'_1 = X \setminus U_1$, $V'_2 = Y \setminus N(U_1)$. Обозначим это паросочетание за $P$. \par
    Рассмотрим множество $U_2 \subset (X \setminus U_1)$. Известно (по условию), что в графе $G$ существует совершенное паросочетание. Это дает нам $|U_2| \leq |N(U_2)|$ и $|N(U_1 \ \cup \ U_2)| \geq |U_1 \ \cup \ U_2| = |U_1| + |U_2|$. По определению множества $U_1$, имеем $|N(U_2) \setminus N(U_1)| \geq |U_2|$. Значит, $|N(U_2) \ \cap \ (Y \setminus N(U_1))| \geq |U_2|$. Получается, что и для графа $G'$ выполнены условия Холла. Значит, в нем действительно существует совершенное паросочетание $P$.\par
    Докажем, теперь, собственно, переход индукции. Итак, по предположению --- в графе $G''(U_1, N(U_1))$ существует не менее $t!$ различных совершенных паросочетаний (степени вершин $U_1$ в подграфе такие же, как во всем графе, условия Холла также выполнены). В исходном графе $G$ также будет не меньше чем $t!$ совершенных паросочетаний: добавим найденное нами совершенное паросочетание $P$ ко всем совершенным паросочетаниям в $G''$.
    
    \item $\forall \text{ непустого } \ U_1 \subset X \text{ (включение строгое) } : |N(U_1)| > |U_1|$.\par
    Рассмотрим произвольную вершину $u \in X$. По условию --- $\operatorname{deg}(u) = n \geq t$. Удалим из графа вершину $u$ и какую-нибудь вершину $v$, такую что: $v \in N(\{u\})$. Заметим, что условия Холла не нарушаются, ведь величина \\ $|N(U_1)|$ уменьшается максимум на один для любого непустого подмножества $U_1 \subset X \setminus \{u\}$. При этом $\forall \ w \in X \setminus \{u\} : \operatorname{deg}(w) \geq (t - 1)$. Значит, в графе $G'(X \setminus \{u\}, Y \setminus \{v\} $ существует не менее $(t - 1)!$ совершенных паросочетаний.\par
    Значит, в исходном графе ($G$) существует не менее $(t - 1)!$ совершенных паросочетаний. Причем, каждое из них содержит $\{u, v\}$, где $v \in N(\{u\})$. \\ Очевидно, $|{\{u, v\} : v \in N(\{u\})}| = n \geq t$. Получим, что в исходном графе $G$ будет уже как минимум $t!$ совершенных паросочетаний.
\end{enumerate}
Переход доказан, а вместе с ним и решена задача.

\end{solution}\par
