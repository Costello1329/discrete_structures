\begin{task}{322}
Пусть 
$n$ — произвольное натуральное число. Пусть $S_1, \dots, S_{n^{2017}}$ — произвольные $n$‑элементные множества. Докажите, что при всех достаточно больших значениях $n$ можно покрасить элементы в красный и синий цвета, так, чтобы в каждом из множеств $S_i$ нашёлся хотя бы один красный и хотя бы один синий элемент.
\end{task}

\begin{solution}
Рассмотрим случайную раскраску. Каждый элемент сделаем красным с вероятностью $1/2$ и синим с вероятностью $1/2$. Событие $H_i$ соответствует наличию в $S_i$ двух цветов. Наше условие: $H_1 \cap H_2 \cap \dots \cap H_{n^{2017}}$.
\begin{multline*}
P(H_1 \cap H_2 \cap \dots \cap H_{n^{2017}}) = 1 - P(\overline{H_1 \cap H_2 \cap \dots \cap S_{n^{2017}}}) =\\ 1 - P(\overline{H_1} \cup \overline{H_2} \cup \dots \cup \overline{H_{n^{2017}}}) \geq 1 - \sum\limits_{i=1}^{{n}^{2017}} P(\overline{H_i}).
\end{multline*}
$Событие \overline{H_i}$ соответствует одноцветности множества $S_i$. 
\[P(\overline{H_i}) = 2\cdot(1/2)^n = 2^{(1-n)}.\]
Получим $P(H_1 \cap H_2 \cap \dots \cap H_{n^{2017}}) \geq 1 - \sum\limits_{i=1}^{{n}^{2017}} 2^{1-n}$. При достаточно больших $n$ выражение положительно, значит требуемая раскраска возможна.\end{solution}