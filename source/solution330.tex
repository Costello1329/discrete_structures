\begin{task}{330}
Отметьте все истинные высказывания.
\begin{enumerate}
\item Лемма Ловаса применима только к наборам событий, независимых в совокупности.
\item Симметричный случай леммы Ловаса выводится из общего случая при помощи математической индукции.
\item Применение леммы Ловаса (в общем случае) требует подбора набора констант.
\item Оценка чисел Рамсея, получаемая при помощи леммы Ловаса, лучше по порядку, чем оценка, полученная «обычным» вероятностным методом.
\item Лемма Ловаса позволяет балансировать между независимостью и маловероятностью событий, которых мы хотели бы избежать.
\item Общий случай леммы Ловаса доказывается с помощью неравенства Маркова.
\end{enumerate}
\end{task}

\begin{solution}
\begin{enumerate}
\item Применение леммы Ловаса (в общем случае) требует подбора набора констант.
\item Лемма Ловаса позволяет балансировать между независимостью и маловероятностью событий, которых мы хотели бы избежать.
\end{enumerate}
\end{solution}
