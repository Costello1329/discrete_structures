\begin{task}{257}
Обозначим через $\eta(H)$ количество различных минимальных вершинных покрытий гиперграфа $H$. Чему равно $\eta(H)$ для $k$-однородного гиперграфа на $n$ вершинах, содержащего ровно $\binom{n}{k} - 1$ гиперрёбер?
\end{task}

\begin{solution}
Данный $k$-однородный гиперграф --- почти полон. Его дополнение содержит лишь одно ребро. Обозначим его за $x$. Понятно, что $V \setminus x$ является покрытием. Оно содержит $n - k$ вершин. Докажем, что меньшее покрытие невозможно.\par
Действительно, возьми мы $n - k - 1$ вершину, осталось бы множество из $k + 1$ вершин, включающее в себя $k + 1$ множеств из $k$ вершин, среди которых будут рёбра. Эти рёбра, очевидно, покрыты не будут.\par
Покажем, что покрытие из $n - k$ единственно. Но ведь очевидно, если выбрать другие $n - k$ вершин, то оставшиеся $k$ образуют непокрытое ребро.\par
Имеем $\eta(H) = 1$.

\end{solution}
