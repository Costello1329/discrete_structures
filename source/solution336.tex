\begin{task}{336}
Подмножества $X_1, \dots , X_n$ и $Y_1, \dots, Y_n$ некоторого $N$-элементного множества таковы, что $X_i$ пересекается с $Y_j$ по пяти элементам при $i = j$ и по четырём элементам иначе. Докажите, что $n \leq N$.
\end{task}

\begin{solution}
Назовем данное нам $N$-элементное множество множеством $A$ с элементами $a_i$. Рассмотрим две матрицы:
\begin{gather*}
    M_1 : {M_1}_{i,j} = \operatorname{I}(a_j \in X_i); \\
    M_2 : {M_2}_{i,j} = \operatorname{I}(a_j \in Y_i).
\end{gather*}\par
Эти матрицы являются матрицами смежности двудольных графов $G_1$ и $G_2$. В левой доле $G_1$ и $G_2$~---~элементы $a_ j$, а в правой~---~множества $X_i$ и $Y_i$ соответственно.\par
Тогда, как известно, при перемножении матриц смежности двудольного графа~---~получим матрицу $M = M_1 \cdot M_2^T$, такую, что ${M}_{i,j} = \{$кол-во общих элементов $X_i$ и $Y_j$ $\}$. По условию: $M_{i,j} = 4 + \operatorname{I}(i = j)$. Данная квадратная матрица, очевидно, имеет ранг, равный ее размеру $n$. Докажем это: 
\begin{gather*}
    \begin {pmatrix}
    	5& 4& 4& \ldots& 4 \\
    	4& 5& 4& \ldots& 4 \\
    	4& 4& 5& \ldots& 4 \\
    	\vdots& \vdots& \vdots& \ddots& \vdots \\
    	4& 4& 4& \ldots& 5 \\
    \end {pmatrix} \sim
    \begin {pmatrix}
    	5& 4& 4& \ldots& 4 \\
    	-1& 1& 0& \ldots& 0 \\
    	-1& 0& 1& \ldots& 0 \\
    	\vdots& \vdots& \vdots& \ddots& \vdots \\
    	-1& 0& 0& \ldots& 1 \\
    \end {pmatrix} \sim
    \begin {pmatrix}
    	5+(n-1)\cdot4& 0& 0& \ldots& 0 \\
    	-1& 1& 0& \ldots& 0 \\
    	-1& 0& 1& \ldots& 0 \\
    	\vdots& \vdots& \vdots& \ddots& \vdots \\
    	-1& 0& 0& \ldots& 1 \\
    \end {pmatrix} 
\\
  \sim  \begin {pmatrix}
    	1& 0& 0& \ldots& 0 \\
    	-1& 1& 0& \ldots& 0 \\
    	-1& 0& 1& \ldots& 0 \\
    	\vdots& \vdots& \vdots& \ddots& \vdots \\
    	-1& 0& 0& \ldots& 1 \\
    \end {pmatrix} \sim
    \begin {pmatrix}
    	1& 0& 0& \ldots& 0 \\
    	0& 1& 0& \ldots& 0 \\
    	0& 0& 1& \ldots& 0 \\
    	\vdots& \vdots& \vdots& \ddots& \vdots \\
    	0& 0& 0& \ldots& 1 \\
    \end {pmatrix}.
\end{gather*}

\par
Таким образом, $n = \operatorname{rk}(M) = \operatorname{rk}(M_1 \cdot M_2) \leq \min(\operatorname{rk}(M_1), \operatorname{rk}(M_2))$.
В свою очередь, очевидно, что $\operatorname{rk}(M_1) \leq \min(n, N)$ и $\operatorname{rk}(M_2) \leq \min(n, N)$.\par
Отсюда, получим $n \leq N$.

\end{solution}