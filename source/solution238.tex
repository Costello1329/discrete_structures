\begin{task}{238}
На курсе $100$ студентов. Известно, что среди них можно выделить 
$149$ различных пар студентов, которые во время семестра давали друг другу списывать на контрольных. Деканат принял решение отчислить после сессии минимально возможное число студентов, но таким образом, чтобы среди оставшихся студентов не осталось ни одной пары списывающих друг у друга. Докажите, что к следующему семестру на курсе останется не менее $26$ студентов.
\end{task}

\begin{solution}
Возьмем граф $G$ на $100$ вершинах (соответствуют студентам). Ребро $e = {u, v}$ в этом графе будет тогда и только тогда, когда $u$ и $v$ студенты списывали друг у друга. В таком графе $149$ ребер (по условию). Рассмотрим граф $G'$, обратный к $G$. Очевидно: $|E'| = \frac{100\cdot(100 - 1)}{2} - 149 = 4801$.\par
Предположим, что к следующему семестру на курсе останется менее  $26$ студентов. Значит, в графе $G$ нет независимого множества на $26$ вершинах. Значит, в графе $G'$ нет клики на $26$ вершинах. По теореме Турана: $|E'| \leq \binom{26 - 1}{2} \cdot4\cdot4 = \frac{25\cdot24\cdot16}{2} = 4800$ (граф, на котором достигается такая оценка - $25$-дольный $4$-регулярный на $100$ вершинах). Имеем $|E'| < 4801$. Противоречие с условием ($|E'| = 4801$)!
\end{solution}