\begin{task}{337}
\begin{enumerate}
\item Какое наибольшее количество рёбер, согласно теореме Турана, может быть в графе на $2018$ вершинах, не содержащем четырёхвершинных клик? Можно дать ответ в виде формулы.\newline
\item Найдите точное значение числа Заранкевича ${Z}_{1,b}(m,bm)$ для произвольных натуральных $b$ и $m$.\newline
\item Что можно сказать про почти все двудольные графы с равномощными долями: доля таких графов, не содержащих $K_{2,2}$, константная // почти все такие двудольные графы не содержат $K_{2,2}$ в качестве подграфа // почти все такие двудольные графы содержат $K_{2,2}$ в качестве подграфа.
\end{enumerate}
\end{task}

\begin{solution}

\begin{enumerate}
 \item Будем действовать по теореме Турана. Разделим граф на три части, внутри которых не будет ребер: по  $672,\, 673,\, 673$ вершин соответственно. Посчитаем наибольшее возможное кол-во ребер в таком графе: $672\cdot673 \cdot 2 + 673\cdot673 = 1357441$.\par
 \item $m\cdot(b-1)$.\par
 \item Правильный ответ: почти все такие двудольные графы содержат$K_{2,2}$ в качестве подграфа. Это следует из нижней оценки числа Заранкевича.\par
\end{enumerate}
\end{solution}