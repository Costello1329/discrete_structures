\begin{task}{138}
Найдите хотя бы один первообразный корень по модулю $11^{1000}$.
\end{task}

\begin{solution}
Для начала, запишем определение первообразного корня $g$ по модулю $m$:
\begin{equation*}
    \forall a \in \mathbb{Z} \ \text{т.ч.} \operatorname{gcd}(a, m) = 1 \ \exists k \in \mathbb{Z} : g^{k} \equiv a \pmod{m}.
\end{equation*}
Вспомним, что можно воспользоваться следующим утверждением:

\begin{theorem} \label{(g+mx)^(m-1)_prim_roots}
Если $g$ --- первообразный корень по простому модулю $m$, то $\exists \ x \ \forall \alpha \geq 1 : g+m\cdot x$ --- первообразный корень по модулю $m^\alpha$. Подходят именно $x$, такие, что: ${\left( g+m\cdot x \right)}^{m - 1} = 1 + m \cdot y$, где $\operatorname{gcd}(y, m) = 1$.
\end{theorem}

Чтобы воспользоваться этой теоремой \ref{(g+mx)^(m-1)_prim_roots} --- найдем сначала первообразный корень по модулю 11. Докажем, что $2$ --- первообразный корень по модулю 11, перебрав все $a \in \{1,2,3,4,5,6,7,8,9,10\}$:
\begin{enumerate}
    \item $a = 1$: \\
    возьмем $k = 10$:  $2^{10} = 1024 \equiv 1 \pmod{11};$
    \item $a = 2$: \\
    возьмем $k = 1$:  $2^1 = 2 \equiv 2 \pmod{11};$
    \item $a = 3$: \\
    возьмем $k = 8$:  $2^8 = 256 \equiv 3 \pmod{11};$
    \item $a = 4$: \\
    возьмем $k = 2$:  $2^2 = 4 \equiv 4 \pmod{11};$
    \item $a = 5$: \\
    возьмем $k = 4$:  $2^4 = 16 \equiv 5 \pmod{11};$
    \item $a = 6$: \\
    возьмем $k = 9$:  $2^9 = 512 \equiv 6 \pmod{11};$
    \item $a = 7$: \\
    возьмем $k = 7$:  $2^7 = 128 \equiv 7 \pmod{11};$
    \item $a = 8$: \\
    возьмем $k = 3$:  $2^3 = 8 \equiv 8 \pmod{11};$
    \item $a = 9$: \\
    возьмем $k = 6$:  $2^6 = 64 \equiv 9 \pmod{11};$
    \item $a = 10$: \\
    возьмем $k = 5$:  $2^5 = 32 \equiv 10 \pmod{11}.$
\end{enumerate}

Теперь можно воспользоваться теоремой~\ref{(g+mx)^(m-1)_prim_roots}:
\begin{equation*}
    {\left( 2 + 11\cdot x \right)}^{10} = 1 + 11\cdot y, \text{ при} \operatorname{gcd}(y, 11) = 1.
\end{equation*} \par
Подставив $x = 0$, получим:
\begin{equation*}
    {\left( 2 + 11\cdot 0 \right)}^{10} = 1024 = 1 + 11\cdot 93.
\end{equation*} \par
Значит, $y = 93$. Осталось проверить, что $\operatorname{gcd}(93, 11) = 1$:
\begin{align*}
    93 = 8\cdot11 + 5;\\
    11 = 2\cdot5 + 1;\\
    5 = 5\cdot1 + 0;\\
    \operatorname{gcd}(93, 11) = 1.
\end{align*}
Получается, действительно, $g = 2$ --- первообразный корень по модулю $11^{1000}$.
\end{solution}
